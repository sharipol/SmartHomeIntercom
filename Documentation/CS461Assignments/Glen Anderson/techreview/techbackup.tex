\documentclass[onecolumn, draftclsnofoot,10pt, compsoc]{IEEEtran}
\usepackage{graphicx}
\usepackage{url}
\usepackage{setspace}

\usepackage{geometry}
\geometry{textheight=9.5in, textwidth=7in}

% 1. Fill in these details
\def \CapstoneTeamName{			Team 25}
\def \CapstoneTeamNumber{		25}
\def \GroupMemberOne{			Lazar Sharipoff}
\def \GroupMemberTwo{			Jordan Davis}
\def \GroupMemberThree{			Glen Anderson}
\def \CapstoneProjectName{		Smart Home Intercom System}
\def \CapstoneSponsorCompany{	Oregon State EECS}
\def \CapstoneSponsorPerson{		D. Kevin McGrath}

% 2. Uncomment the appropriate line below so that the document type works
\def \DocType{		%Problem Statement
				%Requirements Document
				Technology Review
				%Design Document
				%Progress Report
				}
			
\newcommand{\NameSigPair}[1]{\par
\makebox[2.75in][r]{#1} \hfil 	\makebox[3.25in]{\makebox[2.25in]{\hrulefill} \hfill		\makebox[.75in]{\hrulefill}}
\par\vspace{-12pt} \textit{\tiny\noindent
\makebox[2.75in]{} \hfil		\makebox[3.25in]{\makebox[2.25in][r]{Signature} \hfill	\makebox[.75in][r]{Date}}}}
% 3. If the document is not to be signed, uncomment the RENEWcommand below
\renewcommand{\NameSigPair}[1]{#1}

%%%%%%%%%%%%%%%%%%%%%%%%%%%%%%%%%%%%%%%
\begin{document}
\begin{titlepage}
    \pagenumbering{gobble}
    \begin{singlespace}
    	\includegraphics[height=4cm]{coe_v_spot1}
        \hfill 
        % 4. If you have a logo, use this includegraphics command to put it on the coversheet.
        %\includegraphics[height=4cm]{CompanyLogo}   
        \par\vspace{.2in}
        \centering
        \scshape{
            \huge CS Capstone \DocType \par
            {\large\today}\par
            \vspace{.5in}
            \textbf{\Huge\CapstoneProjectName}\par
            \vfill
            {\large Prepared for}\par
            \Huge \CapstoneSponsorCompany\par
            \vspace{5pt}
            {\Large\NameSigPair{\CapstoneSponsorPerson}\par}
            {\large Prepared by }\par
            Group\CapstoneTeamNumber\par
            % 5. comment out the line below this one if you do not wish to name your team
            %\CapstoneTeamName\par 
            \vspace{5pt}
            {\Large
                \NameSigPair{\GroupMemberOne}\par
                \NameSigPair{\GroupMemberTwo}\par
                \NameSigPair{\GroupMemberThree}\par
            }
            \vspace{20pt}
        }
        \begin{abstract}
        % 6. Fill in your abstract    
        The Smart Home Intercom System project will require implementing features that rely on various technologies.
	This document aids in understanding which technologies will be used to implement this project by discussing three topics: audio I/O hardware, data management systems, and person detection. After reviewing and analyzing options for each, this document provides a recommendation on which technologies to use.

	\end{abstract}     
    \end{singlespace}
\end{titlepage}
\newpage
\pagenumbering{arabic}
\tableofcontents
% 7. uncomment this (if applicable). Consider adding a page break.
%\listoffigures
%\listoftables
\clearpage

% 8. now you write!
\section{Audio I/O Hardware}
\subsection{Overview}
Each node in the Smart Home Intercom System will need to be able to get audio input from a user, stream it to another device, and output the audio. Since the Raspberry Pi 3 does not have a built-in audio I/O, additional hardware will be required to satisfy this requirement. 
\subsection{Criteria}
Selecting which audio I/O hardware to use will depend on several criteria, including quality, cost, and ease of configuration. 

\subsection{Potential Choices}
\subsubsection{PI-DAC+ with PI-AMP+}
This configuration would support high quality audio I/O with the Raspberry Pi 3 and require minimal setup [1]. It is also built to protect the hardware it runs on, so even if this component failed the rest of the system should remain intact. Furthermore, it is better documented for the Raspberry Pi 3 than the other listed options, making setup and troubleshooting easier if related problems are encountered. However, it is significantly more expensive than the other options. 
\subsubsection{Cirrus Logic Audio Card}
Like the PI-DAC+ with PI-AMP+, this option would cover audio I/O with high quality. However, the compatibility between this card and the Raspberry Pi 3 is questionable since it was designed for the Raspberry Pi A+ and B+ [2]. However, it can be configured to work with the Raspberry Pi 3 [3]. This option is less expensive but would still cost substantially more than using the onboard audio card, and could require more configuration than either of the other options depending on compatibility issues. Still, this configuration is documented and this project would not be solving the problem for the first time. 

\subsubsection{Onboard audio card}
The onboard audio card on the Raspberry Pi 3 would support audio I/O, but with very low quality [4]. For a final product, this quality of audio would not be acceptable but it may serve the purposes of a prototype if cost becomes a major factor. This would also require no additional configuration, and the built in USB ports could be used for a microphone and speakers. 

\subsection{Discussion}
The best quality option for the least cost reviewed in this document is the Cirrus Logic Audio Card. Since audio I/O is a core piece of this project and the software libraries needed for this project are free, this is a reasonable expenditure. However, the setup could prove difficult if compatibility issues arise. The PI-DAC+ with the PI-AMP+ would offer similar quality and functionality but at a higher cost, but are well documented and fully compatible with the Raspberry Pi 3 (without additional configuration). While it would make setup minimal and contribute no additional costs beyond a microphone and speaker, the built-in audio card for the Raspberry Pi 3 has poor quality compared to the other options. 

\subsection{Conclusion}
Due to its higher quality, cost-effectiveness, and reasonable ease of configuration, the Cirrus Logic Audio card is the best option for audio I/O for this project. It will allow high quality audio I/O which is integral to the Smart Home Intercom System.





























\end{document}
